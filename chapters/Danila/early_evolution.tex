\section{Early Evolution}

Motivated by the popularity of the graphics pipepline, graphics accelerators
in the ??s largely implemented a fixed pipeline in hardware. This section
traces the evolution of this fixed design into one that supports general computation.
This section focuses primarily on Nvidia and their role during this time.

With the design of graphics accelerators heavily coupled to the pipeline,
one critical difference between accelerators was how much of the
graphics pipeline did the accelerator actually support.

Until ??, graphics accelerators did not support the acceleration of
\textit{Transformation} and \textit{Lighting} sub-stages (of the
\textit{Geometry stage}) of the pipeline.
This meant that the CPU would do the \textit{Application} stage and the
the \textit{Transformation} and \textit{Lighting} sub-stages before passing
the resulting vertices to the accelerator to do the rest. Examples of accelerators
that did this include ??.

Nvidia was the first ?? company to put the entire \textit{Geometry} stage
onto the accelerator, allowing the CPU to focus on the \textit{Application}
stage. This meant more high fidelity simulation.


