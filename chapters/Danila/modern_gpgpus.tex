\section{Modern GPGPUs}

This section will cover recent developments in GPGPUs.

\subsection{Reality of Warps}

Recall the Fermi SM diagram in Figure ??. We have two sets of 16 SPs corresponding
two two warp schedulers, giving us 16 SPs per warp.

Imagining warps as 32 cores executing in lock-step is useful in explaining things,
but it isn't realistic. In reality, warp execution takes two cycles as shown
in Figure ??.

This is done because ??

\subsection{Nvidia Developments}

\textbf{Kepler:} This is the first Nvidia GPU to introduce a compiler generate control instruction.
This is inserted every 7 instructions. One byte of the 64-bit instruction indicates
that this is a control instruction. The other 7 bytes encode information for the
next 7 instructions \cite{chipsandcheeseInsideKepler}.

This control instruction encodes latency information, preventing write-after-read hazards
from occuring between fixed-latency instructions \cite{chipsandcheeseInsideKepler}.

In addition, these instructions encode whether or not instructions can be dual-issued \cite{chipsandcheeseInsideKepler}.

\textbf{Maxwell:} This architecture increases the frequency of the control word,
inserting it every 3 instructions instead of every 7. 

Maxwell control instructions include information pertaining to variable latency
instructions. This works by assigning destination registers of these instructions
to a barrier in hardware. Instructions that then have those
registers as operands must wait until the barriers clear up. This means
that Maxwell needs less scoreboarding hardware \cite{chipsandcheeseMaxwellNvidias}.

This architecture also introduced 
