\chapter{Introduction - \textit{Ayrton Chilibeck}}

My portion of this report focused on the GPU memory system. Through the development of this report, the presentation and the GPU assignment I learned that the struggles of using the GPU memory hierarchy effectively is vastly different than the use of the CPU hierarchy. This part of the document will address those differences as well as clear up some fundamental questions about how memory works in the first place. 

We will begin by addressing the construction of memory cells at a transistor level. This will encompass discussion around both the structure of DRAM and SRAM as well as the underlying structure of the transistor itself. We then proceed to the difference in purpose between the GPU and CPU memory systems, as well as the tradeoffs associated with either one. We will discuss differences in design principles underlying both CPU and GPU compute units as well as the varying design of DRAM systems associated with each system. We then discuss the architecture of the GPGPU according to the latest documentation available from both NVIDIA and AMD. This is accompanied by an exploration of the sub-structures contained within each discrete unit of the GPU and how they interact. We also discuss novel architecture decisions and advances in the literature over the last several years and follow this by several worked examples of memory accesses in the GPU memory system. The next section covers the need for memory coherence at the local memory level in certain scientific applications and cache coherence protocols required for such computations. Finally we address novel architectural problems related to the development of heterogenous systems (where both the CPU and GPU share a single die) and how such problems are overcome in the literature and the (admittedly sparse) technical documentation available from AMD, Intel, NVIDIA and Qualcomm.

The most impactful part of this assignment was the chance to spend a significant amount of time refining my understanding of memory systems. I hope to impart some of what I learned in this document and expose certain areas of interest for the reader to explore further.
